\documentclass[12pt,letterpaper]{article}

%Trennungsregeln etc.
\usepackage[english]{babel}

%Schriftart
%see http://www.tug.dk/FontCatalogue/ for more
%Ideen: baskervald
%for working: arev
\usepackage{baskervald}

%Sonderzeichenein-/ausgabe (http://tex.stackexchange.com/questions/44694/fontenc-vs-inputenc for questions)
\usepackage[utf8]{inputenc}
\usepackage[T1]{fontenc}

%Zeilenabstand, Optionen: singlespacing, onehalfspacing, doublespacing
\usepackage[onehalfspacing]{setspace}

%Seitenränder, hier oneside
\usepackage[left=2.5cm,right=2.5cm,top=2.5cm,bottom=2.5cm]{geometry}

%Anführungszeichen (for help see CTAN-package-page:  http://ftp.gwdg.de/pub/ctan/macros/latex/contrib/csquotes/csquotes.pdf)
\usepackage[strict=true,autostyle=true,german=quotes]{csquotes}

%Einrücklänge der ersten Zeile eines neuen Absatzes, standardmäßig drin
\setlength{\parindent}{1cm}

%%Chapter/Section-Überschriften (Schriftart, Größe)
%Section-Überschriften
%\setkomafont{section}{\normalfont \bfseries \Large} 
%Inhaltsverzeichnis auch mit Serifen
%\setkomafont{disposition}{\normalcolor\bfseries} 

\usepackage[ colorlinks=true, urlcolor=blue, linkcolor=green]{hyperref}

%Literaturverzeichnis
%\usepackage[author year]{natbib} %hier muss noch komme zwischen Autor und Jahr weg
%\documentclass{article}% use option titlepage to get the title on a page of its own.
\pagestyle{empty}
\usepackage{blindtext}
\title{Colorful Image Colorization with Tensorflow}
\date{\today}
\author{Sophia Schulze-Weddige \and Malin Spaniol \and Maren Born \\Integrating Artificial Neural Networks with Tensorflow \\Universität Osnabrück}
\begin{document}
\maketitle
\thispagestyle{empty}

\section{Introduction/Motivation}
- we want to built a network that can colorize images
\section{Important background knowledge (including reference to most relevant publications)}
- We started looking at this paper: \\
\url{https://arxiv.org/abs/1603.08511} , \\
- at their best solution
\section{The model and the experiment (MAIN PART). This part should feature code.}
\subsection{Dataset}
- loading large amount of data\\
\url{https://machinelearningmastery.com/how-to-load-large-datasets-from-directories-for-deep-learning-with-keras/}\\
- 
\subsection{Preprocessing}
- image preprocessing documentation\\
\url{https://keras.io/preprocessing/image/#imagedatagenerator-class}\\
- preprocessing via ImageDataGenerator() from keras.prepeocessing.image\\
- takes in traindata, validation data and test data\\
- featurewise-center and featurewise std\\
- classmode: none -> is for predictions\\

\subsection{Layer}
\subsection{Loss-Function}
\section{Visualization and discussion of your results.}
\subsection{Training}
\subsection{Testing}
	
\end{document}
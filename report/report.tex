\documentclass[12pt,letterpaper]{article}

%Trennungsregeln etc.
\usepackage[english]{babel}

%Schriftart
%see http://www.tug.dk/FontCatalogue/ for more
%Ideen: baskervald
%for working: arev
\usepackage{baskervald}

%Sonderzeichenein-/ausgabe (http://tex.stackexchange.com/questions/44694/fontenc-vs-inputenc for questions)
\usepackage[utf8]{inputenc}
\usepackage[T1]{fontenc}

%Zeilenabstand, Optionen: singlespacing, onehalfspacing, doublespacing
\usepackage[onehalfspacing]{setspace}

%Seitenränder, hier oneside
\usepackage[left=2.5cm,right=2.5cm,top=2.5cm,bottom=2.5cm]{geometry}

%Anführungszeichen (for help see CTAN-package-page:  http://ftp.gwdg.de/pub/ctan/macros/latex/contrib/csquotes/csquotes.pdf)
\usepackage[strict=true,autostyle=true,german=quotes]{csquotes}

%Einrücklänge der ersten Zeile eines neuen Absatzes, standardmäßig drin
\setlength{\parindent}{1cm}

%%Chapter/Section-Überschriften (Schriftart, Größe)
%Section-Überschriften
%\setkomafont{section}{\normalfont \bfseries \Large} 
%Inhaltsverzeichnis auch mit Serifen
%\setkomafont{disposition}{\normalcolor\bfseries} 
%insert hyperlinks
\usepackage[colorlinks=true, urlcolor=blue, linkcolor=green]{hyperref}

%insert codesnippets
\usepackage{listings}
\usepackage{xcolor}
\definecolor{deepblue}{rgb}{0,0,0.5}
\definecolor{deepred}{rgb}{0.6,0,0}
\definecolor{deepgreen}{rgb}{0,0.5,0}
\lstset{ 
	language=Python, % choose the language of the code
	commentstyle=\itshape\color{yellow},
	basicstyle=\fontfamily{pcr}\selectfont\footnotesize\color{white},
	otherkeywords={self},
	keywordstyle=\color{violet}\bfseries, % style for keywords
	emph={MyClass,__init__},          % Custom highlighting
	emphstyle=\color{cyan},    % Custom highlighting style
	stringstyle=\color{deepgreen}
	numbers=none, % where to put the line-numbers
	numberstyle=\tiny, % the size of the fonts that are used for the line-numbers     
	backgroundcolor=\color{black},
	showspaces=false, % show spaces adding particular underscores
	showstringspaces=false, % underline spaces within strings
	showtabs=false, % show tabs within strings adding particular underscores
	frame=single, % adds a frame around the code
	tabsize=2, % sets default tabsize to 2 spaces
	rulesepcolor=\color{gray},
	rulecolor=\color{black},
	captionpos=b, % sets the caption-position to bottom
	breaklines=true, % sets automatic line breaking
	breakatwhitespace=false,
	stringstyle=\color{orange}
}


%Literaturverzeichnis
%\usepackage[author year]{natbib} %hier muss noch komme zwischen Autor und Jahr weg
%\documentclass{article}% use option titlepage to get the title on a page of its own.
\pagestyle{empty}
\usepackage{blindtext}
\title{Colorful Image Colorization with Tensorflow}
\date{\today}
\author{Sophia Schulze-Weddige \and Malin Spaniol \and Maren Born \\Integrating Artificial Neural Networks with Tensorflow \\Universität Osnabrück}
\begin{document}
\maketitle
\thispagestyle{empty}

\section{Introduction/Motivation}
- we want to built a network that can colorize images\\
- this project aims to produce colorful images, given a greyscale picture.\\
- transforming greyscale into plausible colors is an easy task for humans\\
- We see a greyscale picture showing a woman playing volleyball at the beach. As we can recognize the scene and the form and relate to it. The sand is yellow, the sea is blue and the ball is white.\\
- But coloring it in life would be a much more difficult task. As we also need to consider different textures, shades and so on. Seeing and imagining things does not make people a proper painter.\\
- Surface structure and the semantics of the scene are necessary to validly color images.\\
- this project aims does not aim to generate the true color for pictures but at least a good and prediction.\\
- aus dem paper: model enough of the statistical dependencies between semantics and the textures of greyscale images and their color versions in order to produce visually compelling results.\\

\section{Important background knowledge (including reference to most relevant publications)}
\begin{lstlisting}
#packages needed
import numpy as np 
import tensorflow as tf 
import cv2
from tensorflow.keras.preprocessing.image import ImageDataGenerator
from tensorflow.keras.models import Sequential
from tensorflow.keras.layers import Conv2D, MaxPooling2D, Flatten, Activation, BatchNormalization
	for loop
'ich bin ein string'
if
return
MyClass,__init__
\end{lstlisting}

\section{``In-text'' listing highlighting}

- We started looking at this paper: \\
\url{https://arxiv.org/abs/1603.08511} , \\
- at their best solution
\section{The model and the experiment (MAIN PART). This part should feature code.}
\subsection{Dataset}
- loading large amount of data\\
\url{https://machinelearningmastery.com/how-to-load-large-datasets-from-directories-for-deep-learning-with-keras/}\\
- 
\subsection{Preprocessing}
- image preprocessing documentation\\
\url{https://keras.io/preprocessing/image/#imagedatagenerator-class}\\
- preprocessing via ImageDataGenerator() from keras.prepeocessing.image\\
- takes in traindata, validation data and test data\\
- featurewise-center and featurewise std\\
- classmode: none -> is for predictions\\

\subsection{Layer}
\subsection{Loss-Function}
\section{Visualization and discussion of your results.}
\subsection{Training}
\subsection{Testing}
	
\end{document}